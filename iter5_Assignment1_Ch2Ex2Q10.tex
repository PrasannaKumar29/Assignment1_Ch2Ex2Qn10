\documentclass[journal,12pt,twocolumn]{IEEEtran}
\usepackage{tikz}
\usepackage{amsmath}
\usepackage{amssymb}
\pagestyle{empty}
\usepackage{setspace}
\usepackage{gensymb}
\singlespacing

\usepackage{amsmath}
\usepackage{amsthm}
\begin{document}
\newcommand{\myvec}[1]{\ensuremath{\begin{pmatrix}#1\end{pmatrix}}}
\newcommand{\cmyvec}[1]{\ensuremath{\begin{pmatrix*}[c]#1\end{pmatrix*}}}
\newcommand{\mydet}[1]{\ensuremath{\begin{vmatrix}#1\end{vmatrix}}}
\newcommand{\proj}[2]{\textbf{proj}_{\vec{#1}}\vec{#2}}
\let\StandardTheFigure\thefigure
\let\vec\mathbf

\title{
Assignment - 1
}
\author{ Prasanna Kumar R - SM21MTECH14001}
\maketitle
\newpage
\bigskip
\bibliographystyle{IEEEtran}
\section*{\textbf{Problem}}
\noindent
\textbf{An isoceles triangle has the extremities of its base at \myvec{2\\5} and \myvec{-2\\2}.
Find the two possible positions of the vertex if its area is 25 sq.units} 


\begin{tikzpicture}
[scale=2,>=stealth,point/.style={draw,circle,fill = black,inner sep=0.5pt},]

%Triangle sides
\def\a{6}
\def\b{5}
\def\c{5}
 
%Coordinates of A
%\def\p{{\a^2+\c^2-\b^2}/{(2*\a)}}
%\def\p{2.25}
%\def\q{{sqrt(\c^2-\p^2)}}

%Labeling points
\node (A) at (0,1.5)[point,label=above right:$A$] {};
\node (B) at (-1.5, 0)[point,label=below left:$B$] {};
\node (C) at (1.5, 0)[point,label=below right:$C$] {};

%Foot of perpendicular

\node (D) at (0,0)[point,label=above right:$D$] {};

%Drawing triangle ABC
\draw (A) -- node[left] {$\textrm{b}$} (B) -- node[below] {$\textrm{}$} (C) -- node[above,xshift=2mm] {$\textrm{b}$} (A);

%Drawing altitude AD
\draw (A) -- node[left] {$\textrm{h}$}(D);

%Drawing and marking angles
%\tkzMarkAngle[fill=orange!40,size=0.5cm,mark=](A,C,B)
%\tkzMarkAngle[fill=orange!40,size=0.4cm,mark=](D,B,A)
%\tkzMarkAngle[fill=green!40,size=0.5cm,mark=](B,A,C)
%\tkzMarkAngle[fill=green!40,size=0.5cm,mark=](C,B,D)
%tkzMarkRightAngle[fill=blue!20,size=.2](A,D,B)
%\tkzMarkRightAngle[fill=blue!20,size=.2](B,D,A)
%\tkzLabelAngle[pos=0.65](A,C,B){$\theta$}
%\tkzLabelAngle[pos=0.65](A,B,D){$\theta$}
%\tkzLabelAngle[pos=1](B,A,C){\rotatebox{-45}{$\alpha = 90\degree -\theta$}}
%\tkzLabelAngle[pos=0.65](C,B,D){$\alpha$}
\end{tikzpicture}
\noindent
\section*{\textbf{Solution}}
\noindent
Let the vertex B be \myvec{2\\5} and vertex C be \myvec{-2\\2}.\\[6pt]
Let the other vertex be A.\\[6pt]
Since ABC is an Isoceles triangle,
\begin{align*}
{\left\|\mathbf{A-B}\right\|}^2 &={\left\|\mathbf{A-C}\right\|}^2 \\[6pt]
{\left\|\mathbf{A}\right\|}^2+{\left\|\mathbf{B}\right\|}^2- 2 \vec{A}^T\vec{B} &= {\left\|\mathbf{A}\right\|}^2+{\left\|\mathbf{C}\right\|}^2- 2 \vec{A}^T\vec{C} \\[6pt]
{\left\|\mathbf{B}\right\|}^2-{\left\|\mathbf{C}\right\|}^2 &= 2 \vec{A}^T\vec{B}-2 \vec{A}^T\vec{C} \\[6pt]
{\left\|\mathbf{B}\right\|}^2-{\left\|\mathbf{C}\right\|}^2 &= 2 \vec{A}^T (\vec{B}-\vec{C}) \\[6pt]
29-8 &= 2. \vec{A}^T \myvec{4\\3} \\[6pt]
21 &= \vec{A}^T \myvec{8\\6}
\end{align*}
If A is chosen as \myvec{x\\y},
\begin{equation}
    8x+6y=21
\end{equation}

\noindent
Given, the area of the triangle= 25 sq.units\\[6pt]
The area of a triangle using vector product is obtained as
\begin{align*}
\frac{1}{2} {\left\|(\mathbf{A-B}) \times (\mathbf{B-C}) \right\|} &=25 \\[6pt]
\frac{1}{2} {\left\|(\mathbf{A-B}) \times \myvec{4\\3} \right\|} &=25 \\[6pt]
{\left\|\myvec{x-2\\y-5} \times \myvec{4\\3} \right\|} &=50 \\[6pt]
3x-4y &= 36 \tag{2}
\end{align*}
Considering negative area for another equation,
\begin{align*}
\frac{1}{2} {\left\|(\mathbf{A-B}) \times (\mathbf{B-C}) \right\|} &= -25 \\[6pt]
\frac{1}{2} {\left\|(\mathbf{A-B}) \times \myvec{4\\3} \right\|} &= -25 \\[6pt]
{\left\|\myvec{x-2\\y-5} \times \myvec{4\\3} \right\|} &= -50 \\[6pt]
-3x+4y &= 64 \tag{3}
\end{align*}
Let us consider the matrices representation of equations (1) and (2),
\begin{align*}
\begin{bmatrix}
3 & -4 \\
8 & 6 
\end{bmatrix}
\vec{X_1}
& =
\begin{bmatrix}
36 \\ 21
\end{bmatrix} \\[6pt]
  \vec{A}\vec{X_1} &=\vec{B} \\[6pt]
  \vec{X_1} &= \vec{A}^{-1}\vec{B} \\[6pt]
\vec{X_1}
& =
\begin{bmatrix}
3 & -4 \\
8 & 6 
\end{bmatrix}^{-1}
\begin{bmatrix}
36 \\ 21
\end{bmatrix}  \\[6pt]
& =
\frac{1}{50}
\begin{bmatrix}
6 & 4 \\
-8 & 3 
\end{bmatrix}
\begin{bmatrix}
36 \\ 21
\end{bmatrix} \\[6pt]
& =
\frac{1}{50}
\begin{bmatrix}
300 \\ -225
\end{bmatrix} \\[6pt]
\vec{X_1}
&=
\begin{bmatrix}
6 \\ -4.5
\end{bmatrix}
\end{align*}
\noindent
Let us consider the matrices representation of equations (1) and (3),
\begin{align*}
\begin{bmatrix}
-3 & 4 \\
8 & 6 
\end{bmatrix}
\vec{X_2}
& =
\begin{bmatrix}
64 \\ 21
\end{bmatrix} \\[6pt]
\vec{A}\vec{X_2} &=\vec{B} \\[6pt]
\vec{X_2} &= \vec{A}^{-1}\vec{B} \\[6pt]
\vec{X_2}
& =
\begin{bmatrix}
-3 & 4 \\
8 & 6 
\end{bmatrix}^{-1}
\begin{bmatrix}
64 \\ 21
\end{bmatrix} \\[6pt]
& =
\frac{1}{-50}
\begin{bmatrix}
300 \\ -575
\end{bmatrix} \\[6pt]
\vec{X_2}
&=
\begin{bmatrix}
-6 \\ 11.5
\end{bmatrix}
\end{align*} \\[6pt]
\noindent
$\therefore$ \textbf{The two possible vertex are $(6,-4.5)$ and $(-6,11.5)$}
\end{document}